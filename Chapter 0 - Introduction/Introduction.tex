% Options for packages loaded elsewhere
\PassOptionsToPackage{unicode}{hyperref}
\PassOptionsToPackage{hyphens}{url}
\PassOptionsToPackage{dvipsnames,svgnames,x11names}{xcolor}
%
\documentclass[
  ignorenonframetext,
  aspectratio=169]{beamer}
\usepackage{pgfpages}
\setbeamertemplate{caption}[numbered]
\setbeamertemplate{caption label separator}{: }
\setbeamercolor{caption name}{fg=normal text.fg}
\beamertemplatenavigationsymbolsempty
% Prevent slide breaks in the middle of a paragraph
\widowpenalties 1 10000
\raggedbottom
\setbeamertemplate{part page}{
  \centering
  \begin{beamercolorbox}[sep=16pt,center]{part title}
    \usebeamerfont{part title}\insertpart\par
  \end{beamercolorbox}
}
\setbeamertemplate{section page}{
  \centering
  \begin{beamercolorbox}[sep=12pt,center]{part title}
    \usebeamerfont{section title}\insertsection\par
  \end{beamercolorbox}
}
\setbeamertemplate{subsection page}{
  \centering
  \begin{beamercolorbox}[sep=8pt,center]{part title}
    \usebeamerfont{subsection title}\insertsubsection\par
  \end{beamercolorbox}
}
\AtBeginPart{
  \frame{\partpage}
}
\AtBeginSection{
  \ifbibliography
  \else
    \frame{\sectionpage}
  \fi
}
\AtBeginSubsection{
  \frame{\subsectionpage}
}
\usepackage{amsmath,amssymb}
\usepackage{iftex}
\ifPDFTeX
  \usepackage[T1]{fontenc}
  \usepackage[utf8]{inputenc}
  \usepackage{textcomp} % provide euro and other symbols
\else % if luatex or xetex
  \usepackage{unicode-math} % this also loads fontspec
  \defaultfontfeatures{Scale=MatchLowercase}
  \defaultfontfeatures[\rmfamily]{Ligatures=TeX,Scale=1}
\fi
\usepackage{lmodern}
\ifPDFTeX\else
  % xetex/luatex font selection
\fi
% Use upquote if available, for straight quotes in verbatim environments
\IfFileExists{upquote.sty}{\usepackage{upquote}}{}
\IfFileExists{microtype.sty}{% use microtype if available
  \usepackage[]{microtype}
  \UseMicrotypeSet[protrusion]{basicmath} % disable protrusion for tt fonts
}{}
\makeatletter
\@ifundefined{KOMAClassName}{% if non-KOMA class
  \IfFileExists{parskip.sty}{%
    \usepackage{parskip}
  }{% else
    \setlength{\parindent}{0pt}
    \setlength{\parskip}{6pt plus 2pt minus 1pt}}
}{% if KOMA class
  \KOMAoptions{parskip=half}}
\makeatother
\usepackage{xcolor}
\newif\ifbibliography
\setlength{\emergencystretch}{3em} % prevent overfull lines
\providecommand{\tightlist}{%
  \setlength{\itemsep}{0pt}\setlength{\parskip}{0pt}}
\setcounter{secnumdepth}{-\maxdimen} % remove section numbering
\usepackage[T1]{fontenc}    % package to select font encodings 
\usepackage{standalone}     % allows to include another tex file and ignore the package loading and the \begin{document} commands 
%\usepackage[latin9]{inputenc}

% A couple of useful packages -------------------------------
\usepackage{xcolor}   % package to get access to more colours, shades, tones, etc
\usepackage{amsthm}   % package to define theorem-like structures
\usepackage{amsmath}  % package to improve math related things
\usepackage{amssymb}  % package for extra mathematical symbols
\usepackage{bm}
\usepackage{babel}    % package to include typographical rules from many languages
\usepackage{graphicx} % package to include graphs, pictures etc
\usepackage{float}    % package to define an place floats like figures more conveniently
\usepackage[caption = false]{subfig}  % package to make captions on figures
\usepackage{pstricks} % package to create colours and other macros
\usepackage{hyperref}
% Graphs packages
\usepackage{tikz, pgfplots}
\pgfplotsset{compat = newest}
\usetikzlibrary{intersections, angles, patterns}
\usepgfplotslibrary{fillbetween}   % For surpluses
\hypersetup{colorlinks = true, linkcolor = SFUblue, urlcolor = SFUblue}

\mode<presentation>    
{	
\usetheme{Luebeck}  % The beamer theme used
%\usetheme{Boadilla}   % Another beamer theme
%\usecolortheme{crane}   % A beamer color set. Looks good, but not as good as SFU's colors
\usecolortheme{SFU}   %The package "beamercolorthemeSFU" is in the same folder as the tex file

\usefonttheme{serif}    
\usefonttheme{structurebold}   % changes the look of slides titles
\setbeamertemplate{blocks}[rounded][shadow = true]  % changes the way \block environments look like
\setbeamertemplate{navigation symbols}{} % Clear navigation symbols at the bottom right corner of the slides
}
% Style -------------------------------------------------------------------

\newrgbcolor{lightblue}{.80 .91 .99}
\newrgbcolor{DarkBlue}{0.2 0.30 0.60}
\newrgbcolor{SeaBlue}{0 0.6 0.8}
\newrgbcolor{DarkRed}{0.8 0 0.2}
\newrgbcolor{pddblue}{.17 .31 .44}
\definecolor{pdlblue}{rgb}{.75,.85,.92}
\definecolor{pdllblue}{rgb}{.9,.95,.98}
\newrgbcolor{FadeBlue}{0.75 0.85 0.92}
\newrgbcolor{SkyBlue}{0.6 0.8 1}

\definecolor{SFUgray}{RGB}{84,88,90}
\definecolor{SFUgold}{RGB}{193,160,30}
\definecolor{SFUred}{RGB}{204,6,51}
\definecolor{SFUblue}{RGB}{28,79,156}

\definecolor{light-gray}{RGB}{236,236,236}
\definecolor{light-gold}{RGB}{240,232,199}
\definecolor{lighter-gold}{RGB}{239,235,222}
\newrgbcolor{darkgreen}{0 0.3 0.3}
\newrgbcolor{medgreen}{0 0.8 0.4}
\newrgbcolor{lightgreen}{0.6 1 0.6}
\newrgbcolor{chartreuse}{0.5 1 0}
\newrgbcolor{charcoal}{0.21 0.27 0.31}
\newrgbcolor{darkcyan}{0 0.55 0.55}

\newrgbcolor{carrot}{0.91 0.41 0.17}

\newrgbcolor{darkolivegreen}{.33333 .41961 .18431}
\newrgbcolor{fadedolivegreen}{.96 .97 0.89}

\newcommand{\indep}{\raisebox{0.05em}{\rotatebox[origin=c]{90}{$\models$}}} % independence of random variables symbol
%-------------------------------------------------------------
\author{}
\date{\vspace{-2.5em}}    % Shows date of compilation, or nothing if using {}
%\begin{document}
%-------------------------------------------------------------
\setbeamertemplate{itemize items}[square]  % have square bullet points 
%\beamerdefaultoverlayspecification{<+->} % to make bullet points appear one by one 
%-----------------------------------------------
%\title[\color{white}Introduction]{Introduction} 
%\author[Chapter 1]{\textbf{Simon Fraser University\\ ECON 220W\\ Spring term 2021}} 


% Will replace the Rmarkdown titles

%\begin{document}

\AtBeginDocument{\title[\hyperlink{outline}{\color{white}Introduction}]{Introduction \\ Definitions, probabilities, distributions, estimators }}  
\AtBeginDocument{\author[Chapter 1]{\textbf{Simon Fraser University\\ ECON 483\\ Summer 2023 } \\ \vspace{0.5cm} \begin{center} \includegraphics[scale = 1]{../ECON logo grey.png} \end{center}}   }

\ifLuaTeX
  \usepackage{selnolig}  % disable illegal ligatures
\fi
\IfFileExists{bookmark.sty}{\usepackage{bookmark}}{\usepackage{hyperref}}
\IfFileExists{xurl.sty}{\usepackage{xurl}}{} % add URL line breaks if available
\urlstyle{same}
\hypersetup{
  pdftitle={Rmarkdown template},
  pdfauthor={Thomas Vigié},
  colorlinks=true,
  linkcolor={SFUblue},
  filecolor={Maroon},
  citecolor={Blue},
  urlcolor={blue},
  pdfcreator={LaTeX via pandoc}}

\title{Rmarkdown template}
\author{Thomas Vigié}
\date{}

\begin{document}
\frame{\titlepage}

\begin{frame}{Disclaimer}
\protect\hypertarget{disclaimer}{}
I do not allow this content to be published without my consent.

All rights reserved \textcopyright  2023 Thomas Vigié
\end{frame}

\begin{frame}{Regression}
\protect\hypertarget{regression}{}
\begin{itemize}
\tightlist
\item
  In this course, we are interested in the following model:
  \begin{equation}
  \bm{Y_i = f(X_{i,1},\,X_{i,2},..., \, X_{i,K}) + u_i }
  \label{regression equation}
  \end{equation} where:

  \begin{itemize}
  \tightlist
  \item
    \(\bm{Y_i}\) is a outcome variable of interest (also called
    \textbf{explained variable} or \textbf{dependent variable})
  \item
    \(\bm{u_i}\) is a \textbf{random disturbance}, also called the
    \textbf{error term}
  \item
    \(\bm{X_{i,1},...,X_{i,K}}\) are the \textbf{explanatory variables}
    or the \textbf{independent variables} or the \textbf{covariates} or
    the \textbf{predictors}
  \end{itemize}
\item
  In words: we are interested in the relationship between the
  explanatory variables and the explained variable
\item
  \(\bm{Y_i}\) and \(\bm{X_{i,1},...,X_{i,K}}\) are random variables. We
  observe realizations of them in the data
\item
  This is called the \textbf{population regression equation}
\end{itemize}
\end{frame}

\begin{frame}{Outline \label{outline}}
\protect\hypertarget{outline}{}
\begin{itemize}
\tightlist
\item
  \hyperlink{rv}{\textbf{Random variables, distributions, moments}}
\item
  \hyperlink{cov}{\textbf{Covariance and correlation}}
\item
  \hyperlink{condrv}{\textbf{Conditional probabilities}},
  \hyperlink{cond exp}{\textbf{expectation}} and
  \hyperlink{cond var}{\textbf{variance}}
\item
  \hyperlink{estimators}{\textbf{Estimators: Definitions and properties}}
\item
  \hyperlink{est vs pred}{\textbf{Estimation vs prediction}}
\item
  \hyperlink{statmod}{\textbf{What is a statistical model}}
\item
  \hyperlink{ML}{\textbf{What is machine learning?}}
\end{itemize}
\end{frame}

\begin{frame}{}
\protect\hypertarget{section}{}
\begin{center} \label{proba}
\LARGE{ \textbf{Random variables and probabilities} } 
\end{center}
\end{frame}

\begin{frame}{Random variables}
\protect\hypertarget{random-variables}{}
\label{rv}

\begin{block}{Definition: Random variable}
A \textbf{random variable} (r.v. for short) is a variable whose value (= realization) depends on outcomes of a \textbf{random phenomenon}. It is generally denoted by upper case letters ($\bm{X}$, $\bm{Y}$) while their realization are denoted by lower case letters ($\bm{x}$, $\bm{y}$).
An \textbf{event} is a set of one or more outcomes involving random variables.
\end{block}

\begin{itemize}
\tightlist
\item
  Take the event ``obtain Heads 5 times out of 7 coin tosses''
\item
  Many outcomes lead to that event:

  \begin{itemize}
  \tightlist
  \item
    5 Heads in a row, then 2 Tails
  \item
    3 Heads, then 2 Tails, then 2 Heads
  \item
    etc (in fact, we could use some combinatorics formulas to figure out
    how many outcomes corresponds to that event)
  \end{itemize}
\end{itemize}
\end{frame}

\begin{frame}{Types of random variables}
\protect\hypertarget{types-of-random-variables}{}
\begin{itemize}
\tightlist
\item
  Random variables can be \textbf{discrete} (0, 1, 2, \ldots) or
  \textbf{continuous} (continuum of values)
\item
  Examples of discrete random variables: Age/Number of kids of the next
  person you meet, number obtained after throwing a die, score obtained
  at a roulette/Black jack table,\ldots{}
\item
  Examples of continuous random variables: Weight/height, temperature,
  score in a course, \ldots{}
\item
  \textbf{Qualitative} variables (Yes/No, Man/Woman, Smoker/Non-smoker,
  car/bicycle/train/bus) can be turned into \textbf{quantitative}
  variables: 1 for Man, 0 for Woman, etc \(\Rightarrow\) back to a
  \textbf{discrete} random variable that a software can understand!
\end{itemize}
\end{frame}

\begin{frame}{Probabilities}
\protect\hypertarget{probabilities}{}
\begin{block}{Definition: Probability}
A \textbf{probability} is a number that represents the proportion of the time an event occurs. For an event \textbf{A}, it is denoted $\bm{\mathbb{P}(A)}$.
\end{block}

Examples:

\begin{itemize}
\tightlist
\item
  Take the event \textbf{A} = ``get a number bigger or equal to 3 when
  throwing a die''. Then \(\bm{\mathbb{P}(A) = 4/6}\) (get a 3, a 4, a 5
  or a 6. 4 possible successes out of 6 possible cases)
\item
  Let \(\bm{X}\) denote the weight of a person and
  \(\bm{A = X>150} \,\textrm{lbs}\). Then \(\bm{\mathbb{P}(A)}\) is the
  probability that a person weighs more than 150 pounds
\end{itemize}
\end{frame}

\begin{frame}{Probability distributions: Discrete random variables}
\protect\hypertarget{probability-distributions-discrete-random-variables}{}
\label{discrete}

\begin{block}{Probability mass function (pmf)}
The \textbf{probability mass function} of a discrete random variable is the list of all possible values of the variable along with the probability they occur.
\end{block}

\begin{itemize}
\tightlist
\item
  Example: The probability mass function of a die throw assigns a
  probability of 1/6 to each number we can obtain: 1, 2, 3, 4, 5, 6
\item
  The probability mass function of a fair coin toss assigns a
  probability of 50\% to each outcome: Heads or Tails
\end{itemize}

\begin{block}{Definition: Cumulative distribution function (cdf)}
The \textbf{cumulative distribution function} is the function that gives the probability that a random variable is lower or equal to a particular value.
\end{block}

\begin{itemize}
\tightlist
\item
  Example: The probability that a die throw returns a number lower or
  equal to 5, i.e.~\(\bm{\mathbb{P}(X \leq 5)}\)
\end{itemize}
\end{frame}

\begin{frame}{Probability distributions: Continuous random variables}
\protect\hypertarget{probability-distributions-continuous-random-variables}{}
\label{continuous}

\begin{itemize}
\tightlist
\item
  Since a continuous variable can take an infinite amount of values,
  \(\bm{\mathbb{P}(X=x) = 0}\) and the previous definition does not make
  sense here

  \begin{block}{Definition: Probability density function (pdf)}
  The \textbf{probability density function (pdf)} is the function representing the \textbf{relative likelihood} of a value a random variable can take vs others. The area under the pdf between two points shows the \textbf{probability} that the random variable equals a value between these two points.
  \end{block}
  \begin{block}{Definition: Cumulative distribution function (cdf)}
  The \textbf{cumulative distribution function (cdf)} is the function that gives the probability that a random variable is lower or equal to a particular value.
  \end{block}
\end{itemize}
\end{frame}

\begin{frame}{Common distributions}
\protect\hypertarget{common-distributions}{}
\small

\begin{itemize}
\tightlist
\item
  \textbf{Uniform distribution}: It gives the same probability to each
  possible number \(\bm{X}\) can take. Exists in the discrete case and
  the continuous case
\item
  \textbf{Bernoulli distribution}: Corresponds to r.v. that can be equal
  to 1 with some probability, and 0 otherwise. Qualitative variables
  turned into binary (0-1) variables are Bernoulli r.v.
\item
  \textbf{Binomial distribution}: Measures the probability of having
  \(\bm{k}\) successes out of \(\bm{n}\) independent trials involving a
  Bernoulli r.v. Example: Getting Heads \(\bm{k = 2}\) times out of
  \(\bm{n=10}\) coin flips
\item
  \textbf{Geometric distribution}: Measures the probability of the first
  success after \(\bm{k}\) attempts
\item
  \textbf{Hypergeometric distribution}: Measures the probability of
  having \(\bm{k}\) successes out of \(\bm{n}\) draws from a finite
  population of size \(\bm{N}\) with \(\bm{K}\) objects corresponding to
  a success (draw \(\bm{k}\) blue balls out of \(\bm{n}\) draws in a urn
  with \(\bm{N}\) balls overall and \(\bm{K}\) blue balls)
\item
  \textbf{Normal distribution}: Continuous distribution that describes
  many variables in real life, and used in many statistical theorems
\end{itemize}
\end{frame}

\begin{frame}{The normal distribution}
\protect\hypertarget{the-normal-distribution}{}
\label{normal}

\begin{itemize}
\tightlist
\item
  Also called \textbf{Gaussian} distribution after Carl Friedrich Gauss
  (genius mathematician)
\item
  Invented by Gauss and Laplace (French) almost simultaneously, but
  independently
\item
  Important parameters: The mean \(\bm{\mu}\) and the variance
  \(\bm{\sigma^2}\). They are enough to characterize the full
  probability distribution so it is denoted
  \(\bm{X \sim \mathcal{N}(\mu, \sigma^2)}\).
\item
  Notable features:

  \begin{itemize}
  \tightlist
  \item
    Symmetric and centered around the mean (the mean is the same as the
    median)
  \item
    The sum of normal r.v. is itself a normal r.v.
  \item
    The ``Bell curve'': Higher probability of being around the mean than
    around the extremes
  \end{itemize}
\item
  Equation of the density function:
  \(\bm{ \phi (x) = \frac{1}{\sqrt{2\pi \sigma ^2}} \exp(-\frac{(x - \mu)^2}{2\sigma ^2} ) }\)
\item
  No need to remember the density function expression
\end{itemize}
\end{frame}

\begin{frame}{Expectation of a random variable}
\protect\hypertarget{expectation-of-a-random-variable}{}
\label{moment}

\begin{itemize}
\tightlist
\item
  In order to understand the distribution of a r.v., it is relevant to
  look at measures of location and dispersion
\item
  These measures are called moments: of order 1 like the expectation,
  order 2 like the variance, and more

  \begin{block}{Definition: Expectation}
  The expectation of $\bm{X}$, denoted $\bm{\mathbb{E}[X]}$ (or often $\bm{\mu_X}$), is the average value $\bm{X}$ takes.
  \begin{itemize}
  \item For a discrete r.v : $\bm{\mathbb{E}[X] = \sum_{i = 1}^{n}x_i\mathbb{P}(X = x_i)}$
  \item For a continuous r.v : $\bm{\mathbb{E}[X] = \int_{-\infty}^{+\infty}xf(x)dx}$ where $\bm{f(x)}$ is the pdf of $\bm{X}$ (no need to know this one)
  \end{itemize}
  \end{block}
\item
  For a 6-faced die throw (\(\bm{X}\) is the obtained number):
  \(\bm{\mathbb{E}[X]=\frac{1}{6}\times 1 + \frac{1}{6}\times 2 + ... + \frac{1}{6}\times 6= 3.5}\)
\item
  Note: When \(\bm{X}\) is discrete, the expectation of \(\bm{X}\) is
  often not equal to any value that \(\bm{X}\) can take (e.g.~a die
  throw)
\end{itemize}
\end{frame}

\begin{frame}{Properties of the expectation}
\protect\hypertarget{properties-of-the-expectation}{}
\begin{block}{Properties of expectations}
Let $\bm{X}$ and $\bm{Y}$ be two random variables, and $\bm{a}$ and $\bm{b}$ be two constants. Then:
\begin{itemize}
\item $\bm{\mathbb{E}[aX] = a\mathbb{E}[X]}$ 
\item $\bm{\mathbb{E}[X + Y] = \mathbb{E}[X] + \mathbb{E}[Y]}$ (linearity)
\item $\bm{\mathbb{E}[a] = a}$ ($\bm{a}$ is not random)
\item In general: $\bm{\mathbb{E}[XY]\neq \mathbb{E}[X]\mathbb{E}[Y]}$
\end{itemize}
\end{block}
\end{frame}

\begin{frame}{Variance of a random variable}
\protect\hypertarget{variance-of-a-random-variable}{}
\begin{block}{Definition: Variance}
The variance of $\bm{X}$, denoted $\bm{\mathbb{V}[X]}$ (often denoted $\bm{\sigma^2}$), is the expectation of the squared deviation from the mean: 
\begin{align*}
\bm{\mathbb{V}[X]} & \bm{\equiv \mathbb{E}[(X - \mathbb{E}[X])^2]}\\
                   & \bm{= \sum_{i = 1}^{n}(x_i-\mathbb{E}[X])^2\mathbb{P}(X = x_i)}
\end{align*}
An \textbf{equivalent} formula is $$\bm{\mathbb{V}[X] \equiv \mathbb{E}[X^2] - (\mathbb{E}[X])^2}$$
and the \textbf{standard deviation} of $\bm{X}$ (often denoted $\bm{\sigma}$) is the square root of the variance.
\end{block}
\end{frame}

\begin{frame}{Properties of the variance}
\protect\hypertarget{properties-of-the-variance}{}
\begin{itemize}
\tightlist
\item
  Because of the square, \(\bm{\mathbb{V}[X]}\) is expressed in the
  square of the unit of \(\bm{X}\). But the standard deviation is
  expressed in the original unit
\item
  The second formula is more useful for r.v. that take a few values
\end{itemize}

\begin{block}{Properties of the variance}
Let $\bm{X}$ and $\bm{Y}$ be two random variables, and $\bm{a}$ and $\bm{b}$ be two constants. Then:
\begin{itemize}
\item $\bm{\mathbb{V}[aX] = a^2\mathbb{V}[X]}$ (the constant can come out, but since the variance is a square, we need to square the constant)
\item $\bm{\mathbb{V}[a] = 0}$ ($\bm{a}$ is a constant, so its average is $\bm{a}$, and it never varies)
\item $\bm{\mathbb{V}[X+Y] = \mathbb{V}[X] + \mathbb{V}[Y] + 2cov(X,Y)}$ 
\end{itemize}
\end{block}
\end{frame}

\begin{frame}{Two random variables}
\protect\hypertarget{two-random-variables}{}
\begin{itemize}
\tightlist
\item
  Often, we do not deal with one random variable but two or more
\item
  Some random variables are \textbf{correlated}, i.e.~the variation of
  one influences the variation of the other (especially in Economics!)
\item
  Examples:

  \begin{itemize}
  \tightlist
  \item
    The proportion of smokers is not the same among men vs women. So the
    probability of meeting a smoker is different if it is a man we meet
    or a woman
  \item
    The proportion of smokers is (definitely!) not the same among
    Canadian vs French people. So the probability of meeting a smoker is
    different if we are in Canada or in France
  \item
    Consumption is not the same between rich and poor households. So
    Consumption is affected by income
  \end{itemize}
\item
  We need to look at measures of \textbf{covariation} between two random
  variables
\end{itemize}
\end{frame}

\begin{frame}{Covariance between two random variables}
\protect\hypertarget{covariance-between-two-random-variables}{}
\label{cov}

\begin{block}{Definition: Covariance}
The \textbf{covariance} between $\bm{X}$ and $\bm{Y}$, denoted $\bm{Cov(X,Y)}$ (or $\bm{\sigma_{XY}}$), is the expectation of the products of the deviations from the mean: 
\begin{align*}
\bm{Cov(X,Y)} &\bm{\equiv \mathbb{E}[(X - \mathbb{E}[X])(Y - \mathbb{E}[Y])]} \\
              &\bm{= \sum_{i = 1}^{n}\sum_{j = 1}^{n}\mathbb{P}(Y = y_i,X=x_j)(x_j - \mathbb{E}[X])(y_i - \mathbb{E}[Y])}
\end{align*}
An \textbf{equivalent} formula is $$\bm{Cov(X,Y) \equiv \mathbb{E}[XY] - \mathbb{E}[X]\mathbb{E}[Y]}$$
\end{block}
\end{frame}

\begin{frame}{Properties of the covariance}
\protect\hypertarget{properties-of-the-covariance}{}
\begin{block}{Properties of the covariance}
Let $\bm{X}$, $\bm{Y}$ and $\bm{Z}$ be three random variables, and $\bm{a}$ and $\bm{b}$ be two constants. Then:
\begin{itemize}
\item A negative covariance means the two random variables evolve in opposite directions
\item A positive covariance means the two random variables evolve in the same direction
\item $\bm{Cov(aX, Y) = a\,Cov(X,Y)}$ 
\item $\bm{Cov(X+Z,Y) = Cov(X,Y)+ Cov(Z,Y)}$ \textbf{(linearity)}
\item $\bm{Cov(aX+Z,bY) = ab\,Cov(X,Y)+ b\,Cov(Z,Y)}$ 
\item $\bm{\mathbb{V}[X+Y] = \mathbb{V}[X] + \mathbb{V}[Y] + 2Cov(X,Y)}$
\item $\bm{Cov(X, X)=\mathbb{V}[X]}$ (check the covariance formula when $\bm{X}$ is in both places!)
\end{itemize}
\end{block}
\end{frame}

\begin{frame}{Correlation between two random variables}
\protect\hypertarget{correlation-between-two-random-variables}{}
\begin{itemize}
\tightlist
\item
  The scale of the covariance is tricky: It is the unit of \(\bm{X}\)
  times the unit of \(\bm{Y}\)
\item
  A ``unit-free'' measure is the \textbf{correlation} between \(\bm{X}\)
  and \(\bm{Y}\)

  \begin{block}{Definition: Correlation}
  The \textbf{correlation} between $\bm{X}$ and $\bm{Y}$, denoted $\bm{Corr(X,Y)}$ (or $\bm{\rho_{XY}}$), is defined as: 
  $$
  \bm{Corr(X,Y) \equiv \frac{Cov(X,Y)}{\sqrt{\mathbb{V}[X]\mathbb{V}[Y]} } }
  $$
  \end{block}
\end{itemize}
\end{frame}

\begin{frame}{Properties of the correlation}
\protect\hypertarget{properties-of-the-correlation}{}
\begin{itemize}
\tightlist
\item
  The units at the top and bottom cancel!
\item
  \(\bm{Corr(X,Y)}\) is always between \textbf{-1} and \textbf{1}
\item
  If \(\bm{Corr(X,Y)=0}\) then we say \(\bm{X}\) and \(\bm{Y}\) are
  \textbf{uncorrelated}
\item
  Note: if \(\bm{Cov(X,Y)=0}\) then \(\bm{Corr(X,Y)=0}\)
\item
  Correlation does not imply causality! The whole difficulty of
  Econometrics is to disentangle the two: Correlation is easy to find
  (just a formula), but causality is more challenging to establish, and
  requires a deeper analysis
\item
  Check out the
  \href{https://www.tylervigen.com/spurious-correlations}{\textbf{spurious correlations website}}
  for weird correlations that have nothing to do with causality
\end{itemize}
\end{frame}

\begin{frame}{}
\protect\hypertarget{section-1}{}
\begin{center} \label{condrv}
\LARGE{ \textbf{Conditional probabilities} } 
\end{center}
\end{frame}

\begin{frame}{Conditional probabilities}
\protect\hypertarget{conditional-probabilities}{}
\begin{block}{Definition: Conditional probability}
The \textbf{conditional probability} of a random variable $\bm{Y}$ \textbf{given} the value of a random variable $\bm{X}$, denoted $\bm{\mathbb{P}(Y=y|X=x)}$ or $\bm{\mathbb{P}(Y|X)}$ is the probability of $\bm{Y}$ taking a value $\bm{y}$ when the value of $\bm{X}$ is fixed at some value $\bm{x}$.
\end{block}

\begin{itemize}
\tightlist
\item
  Example: Consider the world as being composed of circles vs non
  circles, and smokers vs non smokers. We know that \(\bm{20\%}\) of
  people are circles, and \(\bm{80\%}\) are not. Among circles,
  \(\bm{30\%}\) are smokers and among non circles, \(\bm{60\%}\) are
  smokers. Let \(\bm{X}\) be the circle status and \(\bm{Y}\) be the
  smoking status (we could make them equal to \(\bm{0}\) or \(\bm{1}\)).
  We know:

  \begin{itemize}
  \tightlist
  \item
    \(\bm{\mathbb{P}(Y=} \textrm{\textbf{smoker}} \bm{|X=} \textrm{\textbf{circle}} \bm{)=0.3}\)
    and
    \(\bm{\mathbb{P}(Y=} \textrm{\textbf{non smoker}} \bm{|X=} \textrm{\textbf{circle}} \bm{)=0.7}\)
  \item
    \(\bm{\mathbb{P}(Y=} \textrm{\textbf{smoker}} \bm{|X=} \textrm{\textbf{non circle}} \bm{)=0.6}\)
    and
    \(\bm{\mathbb{P}(Y=} \textrm{\textbf{non smoker}} \bm{|X=} \textrm{\textbf{non circle}} \bm{)=0.4}\)
  \item
    \(\bm{\mathbb{P}(X=} \textrm{\textbf{circle}} \bm{)=0.2}\) and
    \(\bm{\mathbb{P}(X=} \textrm{\textbf{non circle}} \bm{)=0.8}\)
  \end{itemize}
\end{itemize}
\end{frame}

\begin{frame}{Conditional probabilities (cont'd)}
\protect\hypertarget{conditional-probabilities-contd}{}
\begin{itemize}
\tightlist
\item
  In words: we know the probability of meeting a circle or non circle,
  and the probability of meeting a smoker \textbf{among (or given)}
  circles or non circles
\item
  What we don't know:

  \begin{itemize}
  \tightlist
  \item
    \(\bm{\mathbb{P}(Y=} \textrm{\textbf{smoker}} \bm{)}\) and
    \(\bm{\mathbb{P}(Y=} \textrm{\textbf{non smoker}} \bm{)}\)
  \item
    \(\bm{\mathbb{P}(X=} \textrm{\textbf{circle}} \bm{|Y=} \textrm{\textbf{smoker}} \bm{)}\)
    and
    \(\bm{\mathbb{P}(X=} \textrm{\textbf{circle}} \bm{|Y=} \textrm{\textbf{non smoker}} \bm{)}\)
  \end{itemize}
\item
  In words: we don't know the probability of meeting a smoker (we don't
  know their total proportion), nor the probability of meeting a circle
  \textbf{among (or given)} smokers or non smokers
\end{itemize}
\end{frame}

\begin{frame}{Conditional expectation}
\protect\hypertarget{conditional-expectation}{}
\label{cond exp}

\begin{itemize}
\tightlist
\item
  The same way we consider conditional probabilities, we can look at
  conditional expectations, i.e.~the expectation of a r.v. when another
  r.v. is fixed
\item
  Example: Let \(\bm{Y}\) be the temperature in Celsius degrees. Let
  \(\bm{X}\) be either Vancouver, or Hasparren (beautiful town in the
  Basque country). The average annual temperatures are
  \(\bm{\mathbb{E}[Y|X=}\textrm{\textbf{Hasparren}}\bm{ ]=13.5^\circ C}\)
  and
  \(\bm{\mathbb{E}[Y|X=}\textrm{\textbf{Vancouver}}\bm{ ]=11^\circ C}\)

  \begin{block}{Definition: Conditional expectation}
  Consider the random variables $\bm{X}$ and $\bm{Y}$. The expectation of $\bm{Y}$ conditional on $\bm{X}$ is defined as:
  $$
  \bm{\mathbb{E}[Y|X=x] = \sum_{i = 1}^{n}y_i\mathbb{P}(Y = y_i|X=x)}
  $$
  \end{block}
\item
  Note: Since it depends on what \(\bm{X}\) is equal to, a conditional
  expectation is \textbf{random}!
\end{itemize}
\end{frame}

\begin{frame}{The law of iterated expectations}
\protect\hypertarget{the-law-of-iterated-expectations}{}
\begin{itemize}
\tightlist
\item
  If Hasparren and Vancouver were the only 2 locations in the world,
  then \(\bm{\mathbb{E}[Y]}\), the world annual average temperature,
  would be the average of
  \(\bm{\mathbb{E}[Y|X=}\textrm{\textbf{Hasparren}}\bm{ ]}\) and
  \(\bm{\mathbb{E}[Y|X=}\textrm{\textbf{Vancouver}} \bm{ ]}\)
\item
  So the expectation of \(\bm{X}\) is an average of the conditional
  expectations!

  \begin{thm*}[Law of iterated expectations]
  The expectation of a random variable is the expectation of conditional expectations:
  $$
  \bm{\mathbb{E}[Y]=\mathbb{E}[\mathbb{E}[Y|X]]}
  $$
  \end{thm*}
\item
  When looking at \(\bm{\mathbb{E}[Y|X]}\), \(\bm{X}\) is fixed. The
  outside expectation is going over the distribution of \(\bm{X}\)
\end{itemize}
\end{frame}

\begin{frame}{Conditional variance}
\protect\hypertarget{conditional-variance}{}
\label{cond var}

\begin{block}{Definition: Conditional variance}
The conditional variance of $\bm{Y}$ given $\bm{X=x}$, denoted $\bm{\mathbb{V}[Y|X]}$, is the expectation of the squared deviation from the mean, conditional on $\bm{X=x}$: 
\begin{align*}
\bm{\mathbb{V}[Y|X]} & \bm{\equiv \mathbb{E}[(X - \mathbb{E}[Y|X])^2|X]}\\
                     & \bm{= \sum_{i = 1}^{n}(x_i-\mathbb{E}[Y|X])^2)\mathbb{P}(Y = y_i|X)}
\end{align*}
An equivalent formula is $$\bm{\mathbb{V}[Y|X]=\mathbb{E}[Y^2|X]- (\mathbb{E}[Y|X])^2  } $$
\end{block}
\end{frame}

\begin{frame}{Population vs sample}
\protect\hypertarget{population-vs-sample}{}
\begin{itemize}
\tightlist
\item
  \(\bm{\mathbb{P}(X=x)}\), \(\bm{\mathbb{E}[X]}\) and
  \(\bm{\mathbb{V}[X]}\) are \textbf{population} values: They are
  \textbf{not} random
\item
  Note: not all random variables have an expectation or a variance!
  Check the Cauchy distribution for an eccentric case
\item
  Any statistic derived from a \textbf{sample} is random. Give me
  another sample, and these measures will be different
\item
  Since sample values (like the sample mean) are random, they can have
  an expectation and their variance is different from 0 (reminder: the
  variance of a nonrandom variable is 0)
\item
  So \(\bm{\bar{X}}\), \(\bm{\hat{p}}\) or \(\bm{\hat{y}_i}\) are
  random: they change with every sample. This is why we care about their
  \textbf{bias}, \textbf{variance} and \textbf{consistency}
\end{itemize}
\end{frame}

\begin{frame}{}
\protect\hypertarget{section-2}{}
\begin{center} \label{estimators}
\LARGE{ \textbf{  Estimators} }
\end{center}
\end{frame}

\begin{frame}{Estimators}
\protect\hypertarget{estimators}{}
\begin{itemize}
\tightlist
\item
  An estimator is a rule to compute an estimate of a given quantity
  given some data
\item
  It produces no more than a guess, educated or not
\item
  If we had access to the population data, we would use it to find, say,
  \(\bm{\mathbb{E}[X]}\)
\item
  Since we only have access to a sample of it, we are going to use an
  estimator to compute an estimate
\item
  \textbf{Ideally}, the bigger the sample, the closer we should get to
  the truth
\item
  Estimates are sample values, so an estimator is random, and hence has
  an \textbf{expectation}
\item
  Computing estimates over different samples should tell us something
  reliable on \textbf{average}
\item
  An estimator being random, we would like to use its distribution to
  infer something about the true value of the parameter of interest
\end{itemize}
\end{frame}

\begin{frame}{Desirable properties of estimators}
\protect\hypertarget{desirable-properties-of-estimators}{}
\begin{block}{Definition: Consistency}
An estimator $\bm{\hat{\theta}}$ of a nonrandom quantity $\bm{\theta}$ is \textbf{consistent} (or \textbf{asymptotically unbiased}) if it converges in probability towards the value it estimates:
\[
\bm{\hat{\theta}} \overset{\mathbb{P}}{\rightarrow}  \bm{\theta}
\]
where $\overset{\mathbb{P}}{\rightarrow}$ denotes convergence in probability, i.e. $\bm{\forall \varepsilon>0}$, $\bm{\mathbb{P}\left( |\hat{\theta} - \theta| > \varepsilon  \right) \rightarrow 0}$ as $\bm{n \rightarrow \infty}$
\end{block}

\begin{block}{Definition: Unbiasedness}
An estimator $\bm{\hat{\theta}}$ of a nonrandom quantity $\bm{\theta}$ is \textbf{unbiased} if on \textbf{average}, it equals the true value of the quantity of interest:
\[
\bm{\mathbb{E}[\, \hat{\theta} \,]= \theta}
\]
\end{block}
\end{frame}

\begin{frame}{Desirable properties of estimators (cont'd)}
\protect\hypertarget{desirable-properties-of-estimators-contd}{}
\begin{block}{Definition: Asymptotic normality}
An estimator $\bm{\hat{\theta}}$ is said to be \textbf{asymptotically normal} if, as the sample size $\bm{n\rightarrow \infty}$:
\[
\bm{{\sqrt{n}(\hat{\theta}) \overset{d}{\rightarrow} \mathcal{N}(\theta, \Omega)}}
\]
where $\bm{\theta}$ and $\bm{\Omega}$ are some nonrandom quantities
\end{block}
\end{frame}

\begin{frame}{Desirable properties of estimators (cont'd)}
\protect\hypertarget{desirable-properties-of-estimators-contd-1}{}
\begin{itemize}
\tightlist
\item
  Many (most) estimators are \textbf{biased}. For some, we can have an
  idea of the bias direction
\item
  But being \textbf{consistent} makes them reliable given the sample
  used is big enough. If an estimator is consistent, the bigger the
  sample, the more accurate the estimator, and the closer the estimate
  is to the true value
\item
  \textbf{Asymptotic normality} allows to make \textbf{inference} about
  the true value of the parameter, i.e.~to draw probabilistic
  conclusions about the true parameter, such as
  \textbf{confidence intervals}
\item
  Both \textbf{consistency} and \textbf{asymptotic normality} rely on
  having a large sample
\end{itemize}
\end{frame}

\begin{frame}{Estimation vs prediction}
\protect\hypertarget{estimation-vs-prediction}{}
\label{est vs pred}

\begin{itemize}
\tightlist
\item
  The whole course is about estimating
  \(\bm{f(x_{i,1},\,x_{i,2},..., \, x_{i,K})}\)
\item
  But what assumptions to make or tools to use depends on whether one
  wants to \textbf{estimate} or \textbf{predict}
\item
  \textbf{Estimation} is about learning about a specific parameter of
  interest. Estimation problems make assumptions about the nature of the
  function \(\bm{f()}\) in \eqref{regression equation} and look for the
  effect of specific variables. Often, adding other variables is meant
  to satisfy assumptions ensuring the accuracy of the estimation method
\item
  \textbf{Prediction} is about guessing, predicting the value of the
  dependent variable given some values of the covariates. So it is about
  making accurate predictions
  \(\bm{\hat{y}_i=\hat{f}(x_{i,1},\,x_{i,2},..., \, x_{i,K})}\).
  Whichever variables are improving predictions will be added, they do
  not have to have a particular relevance theoretically speaking
\item
  The two problems often overlap. But keep in mind what each method
  covered in the course is used for!
\end{itemize}
\end{frame}

\begin{frame}{Conditional expectation and MSE}
\protect\hypertarget{conditional-expectation-and-mse}{}
\begin{itemize}
\item
  In general, one wants to find the best function \(\bm{f()}\), i.e.~the
  one that \textbf{minimizes} the expected distance between \(\bm{Y_i}\)
  and \(\bm{f(X_i)}\)
\item
  That expected distance is called the \textbf{Mean Squared Error}: \[
  \bm{  MSE(f(X_i)) \equiv \mathbb{E} [(Y_i - f(X_i))^2] }
  \]
\item
  By the law of iterated expectations that says
  \(\bm{\mathbb{E}[Y]=\mathbb{E}[\mathbb{E}[Y|X]]}\), we have \[
  \bm{  MSE(f(X_i)) \equiv \mathbb{E} \left[ \mathbb{E} [(Y_i - f(X_i))^2|X_i]\right] }
  \]
\end{itemize}
\end{frame}

\begin{frame}{Conditional expectation and MSE (cont'd)}
\protect\hypertarget{conditional-expectation-and-mse-contd}{}
\begin{itemize}
\tightlist
\item
  From the general variance formula:
  \(\bm{\mathbb{V}[X_i] = \mathbb{E}[X_i^2] - (\mathbb{E}[X_i])^2}\) so
  \(\bm{ \mathbb{E}[X_i^2] = \mathbb{V}[X_i] + (\mathbb{E}[X_i])^2}\)
  and that is the case in the conditional case too
\item
  The MSE can be rewritten \begin{align*}
  \bm{MSE(f(X_i))} & \bm{= \mathbb{E} \left[ \mathbb{E} [(Y_i - f(X_i))^2|X_i]\right] }\\
                 & \bm{= \mathbb{E} \left[ \mathbb{V} [Y_i - f(X_i)|X] + (\mathbb{E}[Y_i - f(X_i)|X])^2 \right] } \\
                 & \bm{= \mathbb{V} [Y_i |X_i] + (\mathbb{E}[Y_i - f(X_i)|X_i])^2}  \\
                 & \bm{= \mathbb{V} [Y_i |X_i] + (\mathbb{E}[Y_i|X_i] - \mathbb{E}[f(X_i)|X_i])^2} \\
                 & \bm{= \mathbb{V} [Y_i |X_i] + (\mathbb{E}[Y_i|X_i] - f(x_i))^2 }\\
  \end{align*}
\end{itemize}
\end{frame}

\begin{frame}{Conditional expectation and MSE (cont'd)}
\protect\hypertarget{conditional-expectation-and-mse-contd-1}{}
\[
\bm{MSE(f(X_i)) = \mathbb{V} [Y_i |X_i] + (\mathbb{E}[Y_i|X_i] - f(x_i))^2 }
\]

\begin{itemize}
\tightlist
\item
  The first term does not contain \(\bm{f(X_i)}\), so it is irrelevant
\item
  If we want to minimize \(\bm{MSE(f(X_i))}\), we need to minimize the
  second term (the first term would vanish in the first order condition)
\item
  That second term is a square, so it is always positive. Minimizing it
  means making it equal to 0: \[
  \bm{f(x_i) = \mathbb{E}[Y_i|X_i]}
  \]
\end{itemize}
\end{frame}

\begin{frame}{Conditional expectation and MSE (cont'd)}
\protect\hypertarget{conditional-expectation-and-mse-contd-2}{}
\begin{itemize}
\item
  The best predictor of \(\bm{Y_i}\) by a function of \(\bm{X_i}\) in
  the MSE sense is the conditional expectation of \(\bm{Y_i}\) given
  \(\bm{X_i}\)!!
\item
  In the linear model, we make the assumption
  \(\bm{\mathbb{E}[Y_i|X_i] = \beta_0 + \beta_1 x_i}\)
\item
  Plug that in \(\bm{MSE(f(X_i))}\) and you get \[
  \bm{MSE(\beta_0, \beta_1)= \mathbb{E}[(Y_i - \beta_0 - \beta_1 X_i)^2]}
  \]
\item
  Rings a bell? It is the population version of \[
  \bm{\frac{1}{n} \sum_{i=1}^n (y_i - \beta_0 - \beta_1 x_i)^2}
  \]
\item
  The objective function the OLS estimates minimize! (see lecture on
  linear models)
\end{itemize}
\end{frame}

\begin{frame}{What is a statistical model?}
\protect\hypertarget{what-is-a-statistical-model}{}
\label{statmod}

\begin{itemize}
\tightlist
\item
  A statistical model is a set of \textbf{assumptions} about the
  distribution of some data or about their relationships
\item
  Example: Consider a pair of equally weighted six-sided dice.
  Underlying assumption: the probability of a die falling on any number
  is equal to \(\bm{1/6}\)
\item
  Example: Consider a pair of weighted six-sided dice. Underlying
  assumption; the probability of a die falling on any number differs
  depending on what face we are thinking of. The probabilities could be
  given to us, or not. And our job could be to figure out the weighting
  of the dice based on a sample of data
\item
  These assumptions will have consequences on the properties of the
  statistical procedures used
\item
  If we wrongly believe that the dice are equally weighted, our
  computations and conclusions will be wrong
\end{itemize}
\end{frame}

\begin{frame}{Model types}
\protect\hypertarget{model-types}{}
3 types of statistical models are distinguished:

\begin{itemize}
\tightlist
\item
  \textbf{Parametric models}: models with a finite number of parameters
  (known or unknown). Parametric models specify the distribution of the
  error term typically (models estimated via
  \textbf{maximum likelihood estimators} are parametric)
\item
  \textbf{Semiparametric models}: models with a finite-dimensional
  component and an infinite-dimensional component. Some assumptions are
  made about the distribution of some random variables, but the
  distribution is not fully specified (typically, linear models are
  semiparametric models as we generally only assume moments for the
  error term, not the full distribution)
\item
  \textbf{Nonparametric models}: models with an infinite number of
  parameters. Minimal assumptions are made for such models. We remain
  agnostic about the shape of \(\bm{f()}\), as we only care about making
  good predictions
  \(\bm{\hat{y}_i=\hat{f}(x_{i,1},\,x_{i,2},..., \, x_{i,K})}\)
\end{itemize}
\end{frame}

\begin{frame}{The role of assumptions in a model}
\protect\hypertarget{the-role-of-assumptions-in-a-model}{}
\begin{itemize}
\tightlist
\item
  The assumptions made have a crucial role for the properties of the
  estimators
\item
  The more restrictive the assumptions, the better the properties of the
  estimator. Is it always worth making restrictive assumptions though?
\item
  Some assumptions are purely technical and needed for estimators to
  have the desired properties. Also hard to verify
\item
  Other assumptions are based on common sense: given the data at hand,
  does it makes sense to assume this or that? There can be evidence in
  favor of an assumption or not
\item
  As a consequence, one estimator might be preferred to another
\end{itemize}
\end{frame}

\begin{frame}{What is machine learning?}
\protect\hypertarget{what-is-machine-learning}{}
\label{ML}

\begin{itemize}
\tightlist
\item
  There are many definitions that differ on several aspects
\item
  Wikipedia says: ``Machine learning (ML) is the study of computer
  algorithms that improve automatically through experience.{[}1{]} It is
  seen as a part of artificial intelligence. Machine learning algorithms
  build a model based on sample data, known as''training data'', in
  order to make predictions or decisions without being explicitly
  programmed to do so''
\item
  1997 by Professor Tom M. Mitchel from Carnegie Mellon University, in
  his famous quote from (1997) ``A computer program is said to learn
  from experience E with respect to some class of tasks T and
  performance measure P if its performance at tasks in T, as measured by
  P, improves with experience E''.
\item
  Self-driving cars use AI systems, the automatic vision system that
  identifies an imminent accident is ML
\end{itemize}
\end{frame}

\begin{frame}{Learning types}
\protect\hypertarget{learning-types}{}
\textbf{Broadly}, 2 types of learning are distinguished:

\begin{itemize}
\tightlist
\item
  \textbf{Supervised learning}: There is a response variable
  \(\bm{Y_i}\) with associated predictors
  \(\bm{X_{i,j}, \, \, j=1,...,K}\). The objective is to estimate a
  relationship between \(\bm{Y_i}\) and the \(\bm{X}\)'s, or predict
  \(\bm{Y_i}\) (Example: \textbf{regression methods})
\item
  \textbf{Unsupervised learning}: There is no response variable, only
  variables \(\bm{X_{i,j}, \, \, j=1,...,K}\). The objective is to
  understand the relationships between the variables or the observations
  (Example: \textbf{cluster analysis} seeks to group observations in
  categories according to the value of the variables)
\end{itemize}
\end{frame}

\begin{frame}{Course outline}
\protect\hypertarget{course-outline}{}
\begin{itemize}
\tightlist
\item
  We will go over many methods that are part of statistical learning
\item
  Linear models are \textbf{semiparametric methods} that allow to
  estimate marginal effects as well as make predictions. But they impose
  constraints on the shape of \(\bm{f()}\)
\item
  \textbf{Nonparametric methods}: Generally used for prediction
  purposes, they assume very little about the shape of \(\bm{f()}\) and
  allow for arbitrary flexibility:

  \begin{itemize}
  \tightlist
  \item
    Nearest neighbors
  \item
    kernel methods
  \item
    Regression/classification trees
  \item
    Neural networks
  \end{itemize}
\item
  \textbf{Unsupervised learning}

  \begin{itemize}
  \tightlist
  \item
    Principal component analysis
  \item
    Clustering
  \end{itemize}
\end{itemize}
\end{frame}

\end{document}
